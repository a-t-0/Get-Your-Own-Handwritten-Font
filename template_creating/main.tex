\documentclass{article}
\usepackage{geometry}
 \geometry{
 a4paper,
 total={180mm,267mm},
 left=-1mm,
 top=5mm,
 }

% suppress page numbers
\pagenumbering{gobble} 

% Generate QR codes in LaTeX
\usepackage{qrcode}  

%Create boxes for qr codes
\usepackage{tikz}
\usetikzlibrary{shapes,snakes}

% Read an array
\usepackage{readarray}
\usepackage{filecontents}

% emulate file
% \begin{filecontents*}{symbol_spec.txt}
% nrOfSymbols = 30
% boxWidth = 3.2
% boxHeight = 2.2
% nrOfBoxesPerLine = 10
% nrOfBoxesPerLineMinOne = 9
% nrOfLinesPerPage = 6
% nrOfLinesPerPageMinOne = 5
% \end{filecontents*}

% \begin{filecontents*}{symbols.txt}
% 1 = .
% 2 = )
% 3 = 3
% 4 = a
% 5 = A
% \end{filecontents*}


% Create a shell for the variable variable names that are read from file
\newcommand\missingcommand[1]{\csname DATA#1\endcsname}
% define separator between variable name and variable value
\readarraysepchar{=}

% Enable latex to loop through variables
\usepackage{forloop}

% TODO: Change encoding to adapt to different languages based on what read_symbol_specs contains
\usepackage[utf8x]{inputenc}

\newcommand\customfont[1]{{\usefont{T1}{custom}{m}{n} #1 }}


\begin{document}
% specify the file to be read the file
\readdef{symbol_spec.txt}{\data}
\readarray*\data\MyDat[-,2] % nr of columns in file I presume

%\MyDatROWS{} variable contains rows of data read.

% Read lines of the symbol_spec.txt into variables and initialize values
\newcounter{nrOfLinesSymbolSpec}
\setcounter{nrOfLinesSymbolSpec}{0}%
\whiledo{\value{nrOfLinesSymbolSpec} < \MyDatROWS}{%
  \stepcounter{nrOfLinesSymbolSpec}%
  \expandafter\xdef\csname DATA\MyDat[\arabic{nrOfLinesSymbolSpec},1]\endcsname{%
    \MyDat[\arabic{nrOfLinesSymbolSpec},2]}%
}

% TODO: verify whether all variable names contain an integer or float value that is non zero. Display error message indicating failing value otherswise.
\begin{itemize}
    \item This are the specifications of the qr dimensions and amount of qr images/symbols.
    \item \missingcommand{nrOfSymbols}
    \item \missingcommand{boxWidth}
    \item \missingcommand{boxHeight}
    \item \missingcommand{nrOfBoxesPerLine}
    \item \missingcommand{nrOfBoxesPerLineMinOne}
    \item \missingcommand{nrOfLinesPerPage}
    \item \missingcommand{nrOfLinesPerPageMinOne}
\end{itemize}

% read symbols

% specify the file to be read the file
\readdef{symbols.txt}{\data}
\readarray*\data\MyDat[-,2] % nr of columns in file I presume

%\MyDatROWS{} variable contains rows of data read.

% Read lines of the symbol_spec.txt into variables and initialize values
\newcounter{nrOfLinesSymbols}
\setcounter{nrOfLinesSymbols}{0}%
\whiledo{\value{nrOfLinesSymbols} < \MyDatROWS}{%
  \stepcounter{nrOfLinesSymbols}%
  \expandafter\xdef\csname DATA\MyDat[\arabic{nrOfLinesSymbols},1]\endcsname{%
    \MyDat[\arabic{nrOfLinesSymbols},2]}%
}

% TODO: verify whether all variable names contain an integer or float value that is non zero. Display error message indicating failing value otherswise.
\begin{itemize}
    \item These are the symbols that are read from file
    \item \missingcommand{1}
    \item \missingcommand{2}
    \item \missingcommand{3}
    \item \missingcommand{4}
    \item \missingcommand{5}
\end{itemize}


% \newcounter{ct}
% ...
% \forloop{ct}{1}{\value{ct} < \missingcommand{nrOfBoxesPerLine}}%
% {%
%   \thect\
  
% }

% read symbols
\newcounter{ct1}
\newcounter{ct2}
\newcounter{ct3}
\newcounter{ct4}

%loop through lines
\newpage
\hspace{-1.1em} % SHift top row left to prevent some section indentation
\noindent\forloop{ct4}{1}{\value{ct4} < \missingcommand{nrOfLinesPerPage}}{
% loop through columns
\begin{tikzpicture}[scale=2][H]
    \tikzstyle{ann} = [draw=none,fill=none,right]
    \matrix[nodes={draw, ultra thin, fill=white},
        %row sep=-\missingcommand{boxWidth}*0.09em,column
        row sep=-3pt,column sep=-\missingcommand{boxWidth}*0.0em] {
    % \hspace{-14em}   
    % create square with symbol in it.
    \forloop{ct1}{1}{\value{ct1} < \missingcommand{nrOfBoxesPerLine}}{
    \node[rectangle,minimum height=\missingcommand{boxWidth}em,minimum width=\missingcommand{boxWidth}em] {.}; \pgfmatrixnextcell
    }\\
    
    % \hspace{-14em}
    % create empty square
    \forloop{ct2}{1}{\value{ct2} < \missingcommand{nrOfBoxesPerLine}}{
    \node[rectangle,minimum height=\missingcommand{boxWidth}em,minimum width=\missingcommand{boxWidth}em] {}; \pgfmatrixnextcell
    }\\
    
    % \hspace{-14em}
    % Create qr Codes
    \forloop{ct3}{1}{\value{ct3} < \missingcommand{nrOfBoxesPerLineMinOne}}{
    \node[draw=none,fill=none] {\qrcode[height=\missingcommand{boxWidth}em]{https://opensource.org/about} }; \pgfmatrixnextcell
    }
    
    % \hspace{-14em}
    % manually print last qr code to close without \pgfmatrixnextcell
    \node[draw=none,fill=none] {\qrcode[height=\missingcommand{boxWidth}em]{https://opensource.org/about} };\\
    };
\end{tikzpicture}
} % close loop through lines
\end{document}